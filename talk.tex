\DocumentMetadata{
  tagging=on,
  tagging-setup={math/setup={mathml-SE,mathml-AF}},
  pdfstandard=ua-2,
  pdfstandard=a-4f,
  lang=en-US
}
\documentclass[12pt]{article}
\usepackage[a4paper, left=0cm, right=0cm, top=0cm, bottom=0cm, paperwidth=34.29cm, paperheight=19.29cm]{geometry} 

\usepackage{unicode-math}
\usepackage{rotating}
\usepackage{xcolor}
\usepackage{tikz}
\usepackage{mathtools}
\usepackage{enumitem}
\usepackage{hyperref}
\hypersetup{
pdftitle={Accessible PDF with LaTeX},
pdfauthor={Bryan W. Lewis},
pdfkeywords={Accessibility, LaTeX, PDF},%
}
\usepackage{authblk}
\usepackage{graphicx}
\usepackage{float}
\pagenumbering{gobble}
\setlength{\parindent}{0pt}
\setlength{\parskip}{0.5em}


\newcommand{\strikethroughminipage}[2]{
    \begin{tikzpicture}[baseline=(current bounding box.north)]
        \node[fill=white] (m) {
            \begin{minipage}{#1}
                #2
            \end{minipage}
        };
        \draw[thick,red] (m.south west) -- (m.north east);
        \draw[thick,red] (m.north west) -- (m.south east);
    \end{tikzpicture}
}

\begin{document}

\noindent  \includegraphics[alt={Accessible PDFs with LaTeX for Teachers}, height=0.999\textheight, width=0.999\textwidth]{dummy}

\vspace{-6em}
\begin{flushright}
\Large{\bf Bryan Lewis and Carla Zeigler, Garrett College, January 2026$\,\,\,$}
\end{flushright}


\newpage


\hspace{4em}
\scalebox{1.8}{%
\begin{minipage}{0.5\textwidth}
\vspace{4em}
\section*{Why \LaTeX?}
\vspace{2em}
\begin{itemize}
\item \LaTeX$\,$  is the de facto standard for the communication and publication of math and science.
Chances are you \emph{will} encounter \LaTeX in STEM!
\item \LaTeX$\,$  is widely used in publishing because it clearly separates presentation from content.
\item \LaTeX typesets technical symbols and mathematics beautifully.
\end{itemize}
\end{minipage}
}

\newpage

\hspace{4em}
\scalebox{1.8}{%
\begin{minipage}{0.5\textwidth}
\vspace{4em}
\section*{Why Accessibility?}
\vspace{2em}
\begin{itemize}
\item
\item
\item
\end{itemize}
\end{minipage}
}

\newpage

\hspace{4em}
\scalebox{1.8}{%
\begin{minipage}{0.5\textwidth}
\vspace{4em}
\section*{Accessible PDF Output is (Recently) Possible!}
\vspace{2em}
\begin{itemize}
\item Thanks to \emph{years} of dilligent work by the \LaTeX project developers:
\item[] \url{https://www.latex-project.org}
\item See this presentation at PDF Days 2025 Berlin:
\item[] \url{https://www.latex-project.org/news/2025/10/30/pdfadays}
\item Also check out the many solution posters from the conference:
\item[] \url{https://pdfa.org/the-winning-technical-poster-at-pdf-days-europe-2025}
\item[]
\item Our slides walk you through a few basic steps to get things working.
\end{itemize}
\end{minipage}
}

\newpage

\hspace{4em}
\scalebox{1.8}{%
\begin{minipage}{0.5\textwidth}
\vspace{4em}
\section*{Step 1: Use \texttt{lualatex}}
\vspace{2em}
\begin{itemize}
\item \LaTeX can be processed by many different rendering engines (\texttt{pdflatex}, \texttt{lualatex}, $\ldots$)
\item \texttt{lualatex} is currently the best engine to use for accessibility.
\item On many GNU/Linux systems simply install with:
\item[] \texttt{sudo apt install texlive-luatex}
\end{itemize}
\end{minipage}
}

\newpage

\hspace{4em}
\scalebox{1.8}{%
\begin{minipage}{0.5\textwidth}
\vspace{4em}
\subsection*{Using \texttt{lualatex} with MiKTeX on Macs and Windows Systems}
\begin{enumerate}
\item Install MiKTeX (\url{https://miktex.org/download})
\end{enumerate}
\vspace{2em}
\end{minipage}
}


\newpage

\hspace{4em}
\scalebox{1.8}{%
\begin{minipage}{0.5\textwidth}
\vspace{4em}
\subsection*{Using \texttt{lualatex} with Overleaf}
\vspace{2em}
\end{minipage}
}


\newpage

\hspace{4em}
\scalebox{1.8}{%
\begin{minipage}{0.5\textwidth}
\vspace{4em}
\section*{Step 2: Metadata in your documents}
\vspace{2em}
Add the following to the very top of your document (adjusting \texttt{lang} as needed):\\

\texttt{\textbackslash{DocumentMetadata}\{ \\
\phantom{x}  tagging=on, \\
\phantom{x}  tagging-setup=\{math/setup=\{mathml-SE,mathml-AF\}\}, \\
\phantom{x}  pdfstandard=ua-2, \\
\phantom{x}  pdfstandard=a-4f, \\
\phantom{x}  lang=en-US \\
\}
}\\

This turns on tagging and embeds \emph{two} kinds of mathematical notation markup for widest compatibility with reader software.
\end{minipage}
}


\newpage

\hspace{4em}
\scalebox{1.8}{%
\begin{minipage}{0.5\textwidth}
\vspace{4em}
\subsection*{PDF document metadata}
\vspace{2em}
Add the following above \texttt{\textbackslash{begin}\{document\}} (adjusting entries as needed):\\

\texttt{\textbackslash{usepackage\{hyperref\}} \\
\textbackslash{hypersetup}\{\\
\phantom{x} pdftitle=\{Accessible PDF with LaTeX\},\\
\phantom{x} pdfauthor=\{Bryan W. Lewis\},\\
\phantom{x} pdfsubject=\{Accessibility\},\\
\phantom{x} pdfkeywords=\{Accessibility, LaTeX, PDF\},\\
\}
}\\

\end{minipage}
}

\newpage

\hspace{4em}
\scalebox{1.8}{%
\begin{minipage}{0.5\textwidth}
\vspace{4em}
\subsection*{Use the \texttt{unicode-math} math symbol package}
\vspace{2em}
Include the following package in your document preamble:\\

\texttt{\textbackslash{usepackage\{unicode-math\}} \\
}\\

\emph{Avoid} use of the following package if possible (\texttt{unicode-math} may work in its place):

\strikethroughminipage{0.3\textwidth}{
{\texttt{\textbackslash{usepackage}\{amssymb\}}}
}

\end{minipage}
}



\newpage

\hspace{4em}
\scalebox{1.8}{%
\begin{minipage}{0.5\textwidth}
\vspace{4em}
\subsection*{Be sure to add text descriptions to every included image}
\vspace{2em}
Whenever you include an image in your document, be sure to include a text description defined with
\texttt{alt}. For example:\\[1em]

\texttt{\textbackslash{includegraphics}[alt={Accessible PDFs with LaTeX for Teachers}]\{filename\}
}

\end{minipage}
}


\newpage

\hspace{4em}
\scalebox{1.8}{%
\begin{minipage}{0.5\textwidth}
\vspace{10em}
\large{Congratulations! Your \LaTeX-generated PDF documents are now accessible!}

\end{minipage}
}



\newpage

\hspace{4em}
\scalebox{1.8}{%
\begin{minipage}{0.5\textwidth}
\vspace{4em}
\section*{Setting up NVDA reader software with MathCAT}
\vspace{2em}
\begin{itemize}
\item These techniques are all \emph{very} new and not yet uniformly supported, but that's changing quickly.
\item The reader software with the most complete support is NVDA (NV Access) equipped with the MathCAT add-on. It's free, supports audible and braille in more than 50 languages, and works very well!
\item[] https://www.nvaccess.org/download/
\end{itemize}

\end{minipage}
}

\end{document}
