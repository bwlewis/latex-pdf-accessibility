\DocumentMetadata{
  tagging=on,
  tagging-setup={math/setup={mathml-SE,mathml-AF}},
  pdfstandard=ua-2,
  pdfstandard=a-4f,
  lang=en-US
}
\documentclass[12pt]{article}
\usepackage{unicode-math}
\usepackage{graphicx}
\usepackage{hyperref}
\hypersetup{
  pdftitle={Differentials},
  pdfauthor={Bryan W. Lewis},
  pdfkeywords={MAT 190 Calculus 1},%
}
\pagenumbering{gobble}

\setlength{\parindent}{0pt}
\setlength{\parskip}{0.5em}


\begin{document}

The derivative symbol $\frac{dy}{dx}$ is a \emph{function}.
Except when it isn't!

The French mathematician \'Elie Cartan helped to formalize the intuitive idea
that $dx$ and $dy$ represent tiny changes in $x$ and $y$. He called them {\it
differentials} and constructed them using local linear approximation inspired
by the definition of derivative.

\begin{figure}[h!]
  \centering
  \includegraphics[alt={Photograph of Elie Cartan}, width=0.3\textwidth]{cartan}\\
  \small{\textit{\'Elie Cartan's magnificent moustache}}
\end{figure}

Let $y = f(x)$ be a differentiable function with derivative $f'(x)$ (using
Lagrange's notation). Let $dx$ be a variable representing a nonzero real number.
Define $dy$ as:
\[
dy = f'(x)dx.
\]
Note that here, $dy$ is a function of $x$ {\it{\bf{ and}}} of $dx$ since it
depends on both of them. Defined in this way, $dy$ and $dx$ are called
``differentials.''

If you divide each side of the above equation by $dx$ you get:
\[
\dfrac{dy}{dx} = f'(x),
\]
which is just the usual way we write the derivative! Thus, we can think of
$dy/dx$ the usual way---as a function of $x$---or, in differential form we can
think separately of $dy$ (as a function of both $x$ and $dx$) and $dx$ (as a
number).


\end{document}
