\DocumentMetadata{
  tagging=on,
  tagging-setup={math/setup={mathml-SE,mathml-AF}},
  pdfstandard=ua-2,
  pdfstandard=a-4f,
  lang=en-US
}
\documentclass[10pt]{article}
\usepackage{unicode-math}
\usepackage{color}
\usepackage{hyperref}
\hypersetup{
pdftitle={Derivative Notation is Awful},
pdfauthor={Bryan W. Lewis},
pdfkeywords={MAT 190 Calculus 1},%
}
\usepackage{cite}
\usepackage{graphicx}
\usepackage{float}
\usepackage{fancyhdr}
\usepackage{geometry}
\geometry{
  margin=54pt,
  includeheadfoot
}
\pagestyle{fancy}
\renewcommand{\today}{\space\ifcase \month \or January\or February\or March\or April\or May \or June\or July\or August\or September\or October\or November\or December\fi,%
\space \number \year}
\rhead{\today}
\lhead{Garrett College Math 190}
\pagenumbering{gobble}

\floatstyle{ruled}
\setlength{\parindent}{0pt}
\setlength{\parskip}{0.5em}


\begin{document}
\section*{Derivative Notation is Awful}

This idea of differentiation we're using to compute derivatives of functions
comes from the late 1600's. And it shows! There are hundreds of years of crufty
and weird notation we need to at least recognize. This note organizes notation
by approximate date and inventor.

In every example below, assume that $f$ is a real-valued, differentiable
function of a real variable (usually called $x$).

\section*{Lagrange's notation (1749)}
\begin{minipage}[t]{0.75\textwidth}
\vspace{0pt}
You're familiar with this one!  The derivative of $f(x)$ is written $f'(x)$.
There is also an interpretation of the apostrophe (aka ``prime'') as an
\emph{operator} on a function. What I mean is:
\[
\left(f(x)\right)' = f'(x).
\]
Note the subtle difference here. On the left of that last equals sign the
prime symbol is {\it doing something} to the function $f$ to return the thing
on the right side of the equals sign, which represents the derivative itself (another function).
In some very specific sense, the $'$ symbol used on the left
is a kind of function that takes one function as input and returns another as output.\\

The derivative of $f'(x)$---if it exists---is written $f''(x)$, often
called the 2nd derivative. The derivative of $f''(x)$---if it exists---is
written, wait for it, $f'''(x)$.
\end{minipage}
\begin{minipage}[t]{0.25\textwidth}
\vspace{0pt}
$\qquad$\includegraphics[width=0.85\textwidth,alt={Portrait of Lagrange}]{lagrange}
\end{minipage}

Now, guess how the derivative of $f'''(x)$ is written! Give up? It's
$f^{(4)}(x)$. Obviously. Note the parenthesis as a bad attempt to avoid confusion
with $f^4(x)$ which in some books might indicate ``the function $f$ evaluated at $x$
then raised to the fourth power''. Madness!


If you're reading some weirdly Roman-influenced text you might see the
author using Roman numerals (of all things!). So, like, $f^{\textrm{iv}}(x)$,
$f^{\textrm{v}}(x)$, etc.  (without parentheses in the superscripts).


\section*{Leibniz's notation (1684)\cite{jl}}
\begin{minipage}[t]{0.25\textwidth}
\vspace{0pt}
$\qquad$\includegraphics[width=0.75\textwidth,alt={Portrait of Leibniz}]{leibniz}
\begin{center}\vspace{-0.5em}\footnotesize{\it (They were really into\\wigs back then...)}\end{center}
\end{minipage}
\hfill
\begin{minipage}[t]{0.73\textwidth}
\vspace{0pt}
This notation is used just as much as the Lagrange notation above.
If the domain variable is called $x$, then
the derivative {\it operator} is written $\frac{d}{dx}$. Think of this
as a function that takes a differentiable function as its argument and
produces a function as its output. Thus,
\[
\frac{d}{dx}\big(f(x)\big) = \frac{df}{dx}(x).
\]
On the left, we have a function $f$ whose derivative with respect to $x$ is being taken. On the right,
a new function called $\frac{df}{dx}$, the first derivative of $f$.
NOTE(!) that the symbols $\frac{d}{dx}$ and $\frac{df}{dx}$ {\bf are not} fractions---they
are simply symbols for functions. Except, of course, when they aren't (see below).
\end{minipage}

Although this seems janky, it's actually a pretty good notation for many reasons.
In particular, it explicitly indicates the dependent variable which is nice.
Also, let's say we just have the graph of some function with $x$ and $y$ variables.
We want to differentiate this {\it curve} (that is, find its slope at each point).
The derivative of the curve using Leibniz's notation is $\frac{dy}{dx}$.
This has a very natural interpretation in terms of slope: 
\[
\dfrac{dy}{dx}\qquad\textrm{reminds you of}\qquad\dfrac{\textrm{change in }y}{\textrm{change in }x}.
\]

{\it Conceptually}, think of $dy$ as a tiny change in $y$ and $dx$ as a tiny change in $x$.
And, again conceptually, the symbol $\frac{dy}{dx}$ reminds us of the ratio that defines slope.
In fact that is literally almost how Leibniz thought about things, but it took until 1966 until Abraham
Robinson formalized those ideas mathematically\cite{arob}.

The Leibniz notation represents 2nd and higher order derivatives as:
\[
\begin{array}{rcl}
\frac{d}{dx}\big(f(x)\big) &=& \frac{df}{dx}(x)\\[8pt]
\frac{d}{dx}\big(\frac{df}{dx}(x)\big) &=& \frac{d^2f}{dx^2}(x)\\[8pt]
\frac{d}{dx}\left( \frac{d^2f}{dx^2}(x) \right) &=& \frac{d^3f}{dx^3}(x)\\
\vdots && \vdots
\end{array}
\]
Note 
$ \frac{d}{dx}\big(\frac{df}{dx}(x)\big) = \frac{d^2f}{dx^2}(x)$ and {\bf{not}} $\frac{d^2f}{d^2x^2}$, $\,$what you might expect if this were algebra class. Welcome to calculus! Remember $dx$ is a symbol that conceptually means a tiny change in $x$---not $d$ times $x$.


\section*{Newton's notation (1684)\cite{in}}

\begin{minipage}[t]{0.65\textwidth}
\vspace{0pt}

This is the notation Isaac Newton used to write derivatives, which he called
``fluxions.'' It's still used in some books (often when differentiating with respect to time),
but is less common than the Lagrange and Leibniz notation. The table below shows
Lagrange and Leibniz notation on the left, and the equivalent Newton notation
on the right:

\[
\begin{array}{rrcl}
y' &=\,\,\frac{dy}{dx} &=& \dot{y}\\[4pt]
y''&=\frac{d^2y}{dx^2} &=& \ddot{y}\\[4pt]
y'''&=\frac{d^3y}{dx^3} &=& \dddot{y}
\end{array}
\]

\end{minipage}
\begin{minipage}[t]{0.35\textwidth}
\vspace{0pt}
$\qquad$\includegraphics[width=0.95\textwidth,alt={Portrait of Newton}]{newton}
\end{minipage}
Beyond three, the dots get a little silly but Newton persevered with outrageous symbols like:
\includegraphics[width=0.5em,alt={Newton's dot notation}]{dot}. We won't be using this notation in class!

\section*{Differentials (ca 1900)}

Above we said that the symbol $\frac{dy}{dx}$ is a {\it function}---that is
$\frac{dy}{dx} = \frac{dy}{dx}(x)$---except when it isn't.
You can thank the famous French mathematician \'Elie Cartan for that.

Cartan helped to formalize the intuitive idea that $dx$ and $dy$ represent tiny changes in $x$ and $y$.
He called them {\it differentials}---a term that even today has several possible interpretations. One way uses local
linear approximation similar to the definition of derivative.\\

\begin{minipage}[t]{0.35\textwidth}
\vspace{0pt}
$\qquad$\includegraphics[width=0.75\textwidth,alt={Portrait of Cartan}]{cartan}
\end{minipage}
\begin{minipage}[t]{0.65\textwidth}
\vspace{0pt}

Let $y = f(x)$ be a differentiable function with derivative $f'(x)$ (using
Lagrange's notation). Let $dx$ be a variable representing a nonzero real number.
Define $dy$ as:
\[
dy = f'(x)dx.
\]
Note that here, $dy$ is a function of $x$ {\it{\bf{ and}}} of $dx$ since it depends on both of them. Defined
in this way, $dy$ and $dx$ are called ``differentials.''\\

If you divide each side of the above equation by $dx$ you get:
\[
\dfrac{dy}{dx} = f'(x),
\]
which is just the usual way we write the derivative! Thus, we can think of $dy/dx$ the usual way---as a function of $x$---or,
in differential form we can think separately of $dy$ (as a function of both $x$ and $dx$) and $dx$ (as a number).
\end{minipage}

Differentials are useful in at least two important applications: integration using substitution, and in the solution of
some differential equations.


\section*{Arbogast's D-notation (1800) and partial derivatives\cite{la}}

\begin{minipage}[t]{0.25\textwidth}
\vspace{0pt}
$\qquad$\includegraphics[width=0.80\textwidth,alt={Portrait of Arbogast}]{arbogast}
\end{minipage}
\begin{minipage}[t]{0.75\textwidth}
This is a differential operator notation that we {\it will} use a lot, but not
until Calc 3 and differential equations courses. It is widely used in the kinds
of problems encountered there. It is often mis-attributed to Euler. Here is the gist:
\[
\begin{array}{rcrcl}
&& Df &=& \frac{d}{dx}\big(f\big)\\[4pt]
&& (Df)(x) &=& \frac{df}{dx}(x)\\[4pt]
&& D^2f &=& \frac{d^2}{dx^2}\big(f\big)\\[4pt]
f_{x} &=& \partial_{x}f &=& \frac{\partial}{\partial x}\big(f\big)\\[4pt]
f_{xx} &=& \partial_{xx}f &=& \frac{\partial^2}{\partial x^2}\big(f\big)\\[4pt]
\end{array}
\]
\end{minipage}
Those last few are called ``partial'' derivatives that we will spend a lot of time on
in the third semester (Calc 3).

\section*{Appendix: Non-standard numbers and infinitesimals}

\begin{minipage}[t]{0.75\textwidth}
\vspace{0pt}
Abraham Robinson devised a ``non-standard analysis'' that
directly implements the intuitive ideas of Leibniz representing $dx$ and $dy$ as
infinitely tiny quantities\cite{arob}. All he had to do was invent a new kind of number
that is larger than any real number. Why not?\\[-4pt]

We normally assume that the real numbers are unbounded.
Robinson throws this out and invents numbers bigger than any real number.
So if $x$ is any real number and $N$ is a non-standard number then $N$ > $x$.
And, weirdly,
for \emph{any} positive real number $p$:
\[
0 < \dfrac{1}{N} < p
\]

\end{minipage}
\hfill
\begin{minipage}[t]{0.23\textwidth}
\vspace{-4.5em}
\includegraphics[width=\textwidth,alt={Portrait of Robinson}]{robinson}\\
\footnotesize{Mathematician and Dilbert boss look-alike contest winner, Abe Robinson}
\end{minipage}
These things: $1/N$ for a non-standard number $N$, are called \emph{infinitesimals}.
Robinson developed a version of calculus based on this.


Remarkably, he showed the non-standard approach to be essentially identical to
the usual calculus, demonstrating that the assumption that real numbers are
unbounded is not necessary (for calculus, anyway).  Some tricky problems are actually easier to solve
with the non-standard approach. Perhaps Leibniz was ahead of his time!



\begin{thebibliography}{99}
\bibitem{la}Arbogast, Louis F. A.. Du calcul des derivations, 1800.
\bibitem{in}Newton, Isaac. De analysi per aequationes numero terminorum infinitas, 1669.
\bibitem{arob}Robinson, Abraham. Non-standard analysis. North-Holland Publishing Co., Amsterdam, 1966.
\bibitem{jl}Roero, Clara Silvia. ``Gottfried Wilhelm Leibniz, first three papers on the calculus (1684, 1686, 1693).'' Landmark Writings in Western Mathematics 1640-1940. Elsevier Science, 2005. 46-58.
\end{thebibliography}

Photo credits
\begin{enumerate}
\item[Lagrange:] Wikipedia \url{https://commons.wikimedia.org/wiki/File:Lagrange_crop.jpg}
\item[Newton:] Godfrey Kneller, Portrait of Sir Isaac Newton, 1689. From\\ \url{https://exhibitions.lib.cam.ac.uk/linesofthought/artifacts}
\item[Cartan:] Wikipedia \url{https://en.wikipedia.org/wiki/File:Elie_Cartan.jpg}
\item[Robinson:] London Mathematical Society \\\url{https://mathshistory.st-andrews.ac.uk/LMS/robinson_lms_obit.pdf}
\end{enumerate}

\end{document}
