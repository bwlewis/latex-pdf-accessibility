\DocumentMetadata{
  tagging=on,
  tagging-setup={math/setup={mathml-SE,mathml-AF}},
  pdfstandard=ua-2,
  pdfstandard=a-4f,
  lang=en-US
}
\documentclass[12pt]{article}
\usepackage{unicode-math}
\usepackage{graphicx}
\usepackage{hyperref}
\hypersetup{
pdftitle={Differentials},
pdfauthor={Bryan W. Lewis},
pdfkeywords={Maxwell-Farady Equation},%
}
\pagenumbering{gobble}

\setlength{\parindent}{0pt}
\setlength{\parskip}{0.5em}


\begin{document}

\section*{The Maxwell-Faraday Equation}

The equation can be written equivalently as a differential equation or an
integral equation. Here is the differential version:

\[
\nabla\times E = -\dfrac{\partial B}{\partial t}
\]

And here it is in integral form:

\[
\int_{\partial \Sigma} E \cdot d\ell = -\dfrac{d}{dt}\iint_\Sigma B \cdot dA
\]

\end{document}
